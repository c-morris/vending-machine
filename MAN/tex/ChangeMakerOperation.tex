\section{Operation of the Change Making Circuit}

\subsection{Initialization}
\begin{enumerate}
\item{Set enable at low}
\item{Cycle Clock from low to high to low}
\item{Fill machine with bills to make change}
\item{Reset change counters, by cycling switches 1,2, and 3 low to high to low}
\end{enumerate}

\subsection{How to Make Change}
\begin{enumerate}
\item{Feed the value you want to make change of over the data bus}
\item{Make certain clock and enable are low}
\item{Set clock to high, then to Low}
\item{Set enable to high}
\item{Allow clock to cycle}
\item{Read output LEDs as bill to dispense and clock for dispenser}
This prototype emulates support for the bill dispensing machine 
described in this manual.
\end{enumerate}

\subsection{Maintenance}
\begin{itemize}
\item{There are three LED's which mark when their category of  bills is empty
for that category. The LED's are labeled for their value, and are shown on the
specification diagram.}

\item{When the restock change LED's are lit the maintainer must open up the 
machine and fill up the appropriate bill and then reset its counter by 
flipping its switch: low to high to low.}

\item{The 10s have a reset switch at 3. The 5s have 2 as their reset switch.
The 1s have 1 as their reset switch.} 

\item{For clarification to more technical analysis: reset refers to set on
the chain of flip flops, but is stated as reset to improve understanding for 
technicians who lack an electrical or computer engineering background.}

\item{If issues arise where initialization fails try toggling clock while
enable is low.}
\end{itemize}

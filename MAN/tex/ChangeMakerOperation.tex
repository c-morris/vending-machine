\section{Operation of the Change Making Circuit}

\subsection{Initialization}
\begin{enumerate}
\item{Set enable at low}
\item{Fill machine with bills to make change}
\item{Reset change counters, by cycling switch 1 from high to low to high}
\item{Circuit must have spent time in the non-vending state after powering it on}
\end{enumerate}

\subsection{How to Make Change}
\begin{enumerate}
\item{Feed the value you want to make change of over the data bus}
\item{Set Enable to low}
\item{Allow clock to cycle}
\item{Read the two upper output LEDs as bill to dispense and the bottom as the clock for dispenser}
\begin{enumerate}
\item{00 is one}
\item{01 is five}
\item{10 is ten}
\end{enumerate}
This prototype supports some form of bill dispensing machine, which takes a two bit number as a selector input for the type of bill and a clock pulse to know to dispense the bill
\end{enumerate}

\subsection{Maintenance}
\begin{itemize}
\item{There are two LED's which mark when their category of  bills is empty. The LED's are labeled for their value, and are shown on the specification diagram.}

\item{When the restock change LED's are lit the maintainer must open up the machine and fill up the appropriate bill and then reset its counter by 
flipping its switch: low to high to low.}
\item{The reset switch is at position 7.} 

\item{For clarification to more technical analysis: reset refers to set on
the chain of flip flops, but is stated as reset to improve understanding for 
technicians who lack an electrical or computer engineering background.}

\item{If issues arise where initialization fails try toggling clock while
enable is low.}
\end{itemize}

\section{Vending Machines and Honors}
\emph{This section covers the projects development history and the rigorousness of our work.}\\  


This project was a very interesting extension of the normal coursework
for CSE2300W, and was an excellent learning experience. The goal of 
building the circuitry for a vending machine involved large amounts of 
coordination. We initially developed a well defined framework for 
interaction between our circuits, so that we could treat each others 
circuits as black boxes which perform a reliable action. We had 
hoped that this would limit group communication as we would already know 
everything we needed to know about how each other's circuits worked, 
but in the end, we often needed to discuss changes and suggestions. This 
was because our initial predictions for how best to do the 
intersections between circuits proved suboptimal and we needed to develop 
improvements. The failure of our initial design plan resulted from the 
circuit being more difficult to design than the circuits we built in our 
lab sections.\\


We began this project by planning out the use case of our circuit and
arrived upon the interesting idea of having the NSA be our customer. Then 
we moved onto developing the overarching idea of how our vending machine 
would work. The main idea was to split the project into simple subunits, 
which would be evenly distributed. After distributing these tasks, we 
developed requirements for each point of connection between our circuits. 
We continuously improved upon these ideas during the project, discussing 
additions to and removals from the design. Lastly, we developed complete 
functional simulations of our circuit.\\


The circuit gradually developed over the course of the second half of the 
semester. Eventually, we realized that certain parts of our circuit would 
be quite large and it would be optimal to reuse parts of the circuit with
similar functionality. In addition to this, we realized that we could 
reduce the amount of wire used to connect our subunits if we redistributed
some of them. So, on the fourth week of meetings we reassigned the 
number of bills in the machine to the change making circuit, which made the
inventory and change making portion much more self contained. The final 
stage involved assembling our individual components into one vending 
machine. This was the most difficult part of the project, but it was 
gratifying as it facilitated development of teamwork skills far beyond what 
was necessary in the labs. In the normal lab sections we would only talk
with neighbors about the lab assignment's specifications, or particularly 
interesting solutions we came up with. Within the non-honors portion of 
the class we never had to cooperatively develop distinct circuits which 
rely on the correctness of our team member's circuit.\\


This project was also much more difficult than any of the lab work we 
completed during normal meeting times. Our circuit was sequential in 
nature and making functional sequential circuits is a step above what we 
had done in lab, as sequential circuits relies on the timing of specific 
events. New issues arise as false answers, which appear for only a couple, 
nanoseconds will break a sequential logic circuit, while they don't 
affect non-sequential circuits in any human discernible way. The specific 
circuits we had to develop for the vending machine were also 
significantly larger, in terms of scope, than the ones we made in lab. 
This meant that the designing our circuits required that we learn to use 
our whole toolkit of chips to solve our assigned problems. For 
example, Cameron used NAND gates to emulate other gates so that he had 
optimal space usage, as board real estate was rather valuable. \\


Overall we ended learning a lot about circuit design and the difficulties 
of developing circuits which rely heavily on interoperation. We also saw 
how a lack of communication would have lead to extra work. Discussing our
circuits allowed us to recognize when we were repeating each others work, 
and could reuse each other's circuits in ways we had originally slipped 
past us. As a result, we managed to avoid a large amount of work by 
properly planning the project, and by maintaining a high level of 
correspondence.


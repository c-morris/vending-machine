\section{Vending Machines and Honors}
\emph{This section covers the projects development history and the rigorousness of our work.}\\  


This project was a very interesting extension of the normal coursework
for CSE2300W, and was an excellent learning experience. The goal of 
building the circuitry for a vending machine involved large amounts of 
coordination. Despite our best efforts to pre-define all connections, we
found that communication was very important to the proper construction
of our circuit. The actual design of the circuit was also much more 
difficult than the normal circuits we built during lab sections this year.


We began this project by planning out the use case of our circuit and
arrived upon the interesting idea of having the NSA be our customer. Then 
we moved onto developing the overarching idea of how our vending machine 
would work. The main idea was to split the project into as many simple 
subunits as possible, and then distribute the components fairly. It took 
two meetings to develop these subunits where at the end of the second we 
distributed the subunits based on who wanted what. Luke took the change 
making circuitry. Cameron took the primary action of bills in machine 
including adding new bills that are read and subtracting the price of an 
item when it is vended. Brandon took all the inventory and change bills 
in the machine along with the display. After distributing these tasks we 
developed requirements for each point of connection between our circuits. 
This specification was very simple, but really only simplified the design 
phase and we continued modifying these interconnecting points as our 
designs developed.


The project gradually developed over the course of the second half of the 
semester. Eventually we realized that certain parts of our circuit would 
be quite large and it would be optimal to abstract them away. In addition 
to this we found that certain subunits were better localized on other 
parts of the circuit. On the fourth week of meetings we reassigned the 
number of bills in the machine to Luke as determining the amount of 
change to dispense relies on every bit of each bill counting register. 
Reassigning this bit of memory did add complexity to the change making 
circuit, but it reduced the complexity of combining all our subunits 
together. In retrospect, it was a good idea, but it ended up filling 
a large amount of space on the change making circuit board. This project 
facilitated development of teamwork skills far beyond what was necessary 
in the labs. Generally, in the normal lab sections we would communicate 
with neighbors about the lab assignments specifications, or particularly 
interesting solutions. In this project we had to discuss a large amount of
detail about each of our parts.


This project was also much more difficult than any of the lab work we 
completed during normal meeting times. Our circuit was sequential in 
nature and making functional sequential circuits is a step above what we 
had done in lab as sequential circuits rely on the timing of specific 
events and certain issues including momentarily false answers which 
appear for a couple nanoseconds, will break a sequential logic circuit. 
The specific sequential logic circuits we had to develop were also 
significantly larger in terms of scope than the ones we made in lab. 
Our individual contributions were also much larger than normal circuits 
from the weekly labs which made debugging several orders of magnitude 
harder. The final circuits we used also required that we learn to use 
our whole toolkit of possible chips to solve the problem at hand. Cameron 
ended having to actually use NAND gates for their alternate purpose to
save on space within his board, as board real estate was rather valuable. 


Overall we ended learning a lot about circuit design and the difficulties 
of developing circuits which rely on heavy interoperation. We also saw, 
how a lack of communication would have lead to extra work in regards to 
not recognizing when we were repeating each others work, and having to 
redo our own work due to lack of detail in requirements. We managed to 
avoid a large amount of work by properly specifying our requirements 
before beginning work.


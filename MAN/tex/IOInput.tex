\section{Currency Input IO}
The purpose of this section is to Provide extra detail on the specification of IO streams and how they interface with other devices. 

\subsection{Input Details}
\label{selector}
\begin{itemize}
\item \textbf{Clock:} The clock signal tells the circuit to add the value of an accepted bill to the value that is currently stored in memory. In a real vending machine, the clock pulse would be generated when the bill reader accepted an inserted bill as valid. We simulate this using a manually operated two-button clock pulse generator. 
\item $\mathbf{B_{0..5}}$\textbf{:} Together these inputs are a binary representation of the value of the bill with $B_0$ being the least significant bit. In a real vending machine, this number would be generated by the bill reader. We simulate this in our circuit using 6 switches. 
\item \textbf{Selector:} The selector bit switches the ALUs between addition and subtraction while adjusting their $C_n$ bits accordingly. Addition is used for currency input and subtraction is used for vending and making change. The value of this bit is set by the item selection and change making circuits. 
\end{itemize}

\subsection{Output Details}
\label{dollars-in-machine}
\begin{itemize}
\item \textbf{Dollars In Machine:} There are six D flip flops that store the amount of money currently in the machine. On the board, they are numbered as such where FF represents a 74LS74: \\\\
\begin{tabular}{|c|c|c|}
\hline 
4 & FF & 1 \\ 
\hline 
5 & FF & 2 \\ 
\hline 
6 & FF & 3 \\ 
\hline 
\end{tabular} 
\end{itemize}

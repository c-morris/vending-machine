\section{What We Learned}

\subsection{Cameron}
Multiplexers are awesome. When we were first introduced to the concept of a mux it was in lab where we used it as a function generator. Muxes are definitely useful as function generators but they were most valuable to us when we were reusing the same components for different operations. The circuit that I built has two four-bit ALUs (74\_181) at its center. ALUs are useful because they are able to perform many different functions on a single chip, however, this means that the circuit needs to be able to switch between different inputs for different operations. This is where multiplexers come in. Using two quad 2-1 multiplexers I was able to reuse the same ALUs to both add the value of inserted bills to the total and subtract the value of a vended item as well as dispensed change. Furthermore, Luke has an additional mux on his board that allows switching outputs between his change output circuit and Brandon's item counting circuit. Multiplexers were very useful in the development of the vending machine. \\

I also learned how valuable it is to be able to manipulate boolean expressions to use only certain logic gates. Board real estate was in short supply, so choosing to implement functions like in Figure \ref{clock-enable-simplified} saved a lot of space. \\

Agreeing on specifications for the interoperation of our circuits before building them was also key to us succeeding at this project. Although all the specifications really consisted of were hand drawn diagrams that we made during meetings, they were enough that we had almost no issues wiring the protoboards together. This was a point where a lot of things could have gone wrong, but they did not. The importance of this kind of planning is something that I will take away from this project and apply in the future.

\subsection{Luke}
Using Multiplexers to select between distinct input streams was extremely useful. The 2-1 Multiplexer chips let us toggle between streams, and allowed us to feed the output of each board to one ALU, and then toggle between the inputs. This made interoperation significantly easier and allowed us to reuse the two ALUs on one board for all our boards. The three state nature of our device meant that we needed some memory for knowing what state we are in and what state we should go to. This was a surprisingly difficult thing to do and we ended up using an switch to determine that the next clock cycle would do the initialization. The switch would prime a flip flop to toggle on at the same clock tick. So, after the initialization occurs the circuit switches back into its normal running state, and then shuts itself off after the calculation is done. I found this solution to be really interesting and I will use it in future circuit designs.\\


I also learned that it is good to plan out for significantly more expendable material than is expected to be used. If you expect to use 10 feet of wire, but don't really know how much wire you're going to use then you should plan for 20 feet as opposed to allowing yourself to run out. I don't mean buying more than you would need, as that's inefficient, but instead making sure you have a supply source in case your estimates were bad. We had a major issue with running out of wire, and even though I was efficient with my wire usage I had to scrounge wire from a wide variety of people all over the CSE department. This did teach me how useful it is to know people within the same field as you and that it is a good idea to ask for help as you never know who may have the parts you need. In total it really showed me how important the planning phase of an engineering project is. Planning was integral to the final construction phase of our project and as a result of coordination we were able to link all our pieces together with minimal effort, even if I had to hunt down more wire.\\


\subsection{Brandon}
The most difficult portion of building my circuit was ensuring the decrement operation was properly demuxed to the right item. Since there were no demuxes in our kit, one had to be built. Normally, we do not have to account for propagation delay in the circuits we build in lab but because a glitch would be caused by combination logic gates, a solution purely using this would not be feasible. It may suceed in decrementing the intended circuit when the vend clock pulse was sent, but delays may have resulted in temporary 1 glitches decrementing other counters. To fix this, Luke had helped me implement a solution using a quad 2-1 mux. \\

Another issue we debugged was my circuit failing when connected to a teammates. The issue turned out to be variances in the voltage levels supplied separately to our circuit. Once our circuits were jumped together from the same source, the circuits behaved as intended. I initially did not believe this to be the issue when it was proposed to me thinking the voltage difference would have been small enough to be negligible. I would not have solved this on my own, and learned through this project how critical supplied VCC levels can be for a integrated circuit in determining high/low input.

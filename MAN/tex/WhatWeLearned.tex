\section{What We Learned} 

\subsection{Cameron}
Multiplexers are awesome. When we were first introduced to the concept of a mux it was in lab where we used it as a function generator. Muxes are definitely useful as function generators but they were most valuable to us when we were reusing the same components for different operations. The circuit that I built has two four-bit ALUs (74\_181) at its center. ALUs are useful because they are able to perform many different functions on a single chip, however, this means that the circuit needs to be able to switch between different inputs for different operations. This is where multiplexers come in. Using two quad 2-1 multiplexers I was able to reuse the same ALUs to both add the value of inserted bills to the total and subtract the value of a vended item as well as dispensed change. Furthermore, Luke has an additional mux on his board that allows switching outputs between his change output circuit and Brandon's item counting circuit. Multiplexers were very useful in the development of the vending machine. \\

I also learned how valuable it is to be able to manipulate boolean expressions to use only certain logic gates. Board real estate was in short supply, so choosing to implement functions like in Figure \ref{clock-enable-simplified} saved a lot of space. \\

Agreeing on specifications for the interoperation of our circuits before building them was also key to us succeeding at this project. Although all the specifications really consisted of were hand drawn diagrams that we made during meetings, they were enough that we had almost no issues wiring the protoboards together. This was a point where a lot of things could have gone wrong, but they did not. The importance of this kind of planning is something that I will take away from this project and apply in the future. 

\subsection{Luke} 
In the future I should get more wire so that I don't run out.
\subsection{Brandon}
